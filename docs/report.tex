\documentclass[journal=jacsat, manuscript=article]{achemso}

\usepackage[version=3]{mhchem}
\usepackage{courier}

\newcommand*{\mycommand}[1]{\texttt{\emph{#1}}}

\author{Sabrina Jiang}
\author{Will Sherwood}
\author{Kasra Sadeghi}

\title{Backbone Compiler}

\begin{document}

\begin{abstract}

The goal of this project is two-fold: firstly, design a programming language that has enough features
to run system calls and do basic programming tasks. Secondly, the programming language should be designed with the intent of it being able to compile itself. 

We completed the first task by creating Backbone, a lisp-like language that has recursive functions, let bindings, arithmetic and more basic features. 

To complete the second task, we wrote a compiler for Backbone in C to translate it directly into LLVM bytecode. This was created in a way to make it easy to translate into Backbone later, once the language had been developed more. This would result in a self-bootstrapping compiler in Backbone.

\end{abstract}

\section{Introduction}

Our language, Backbone, was designed to be able to eventually compile itself. We worked towards this goal by first writing a C compiler in a way that allowed us to easily translate into Backbone later. This meant that at times, we couldn't use convenient C shortcuts to write the compiler, since it was likely that Backbone wouldn't be able to support such complex features.

Despite failing to complete the Backbone compiler written in Backbone for this project, we were able to flesh out a parser, LLVM generator, and more in the amount of time we had. Additionally, having these written in C was an advantage because we wanted Backbone to be inspired in part by C.

We chose to compile Backbone into LLVM, a widely used low-level language that was created to be flexible and reliable. LLVM started as a research project and has now become a library of modular technologies intended for use in a compiler, perfect for our needs in this project. Also, we use Clang and CMake to compile our source code, both of which are respectable C compilers. Clang in particular was convenient because 

\section{Language}

The motivation behind the design of the Backbone language was to be
simple enough to write a simple compiler for it, but also have enough
features to be able to write a backbone compiler in the backbone language.
Through many design choices, we came to the language syntax detailed below. 

After some trial and error, we concluded that the best way to make syntax decisions was to first write the target feature in LLVM, examine the syntax, and then reverse-engineer the corresponding Backbone syntax. For example, while creating function definitions, the original syntax was as follows: 
$$\texttt{(def main (paramlist (param argc i32) (param argv i8**))}$$

However, after examining the LLVM and discussing which keywords were necessary, we decided on the following format instead:
$$\texttt{(def main (params (argc i32) (argv i8**)) i32}$$

This new format was much better. Firstly, it provides everything needed to generate the LLVM. We realized the return type was required knowledge, and added that to the definition. Secondly, it eliminates some of the unnecessary keywords.

The rest of our syntax was decided in a similar manner. The biggest consideration was balancing between inferring information, such as types of identifiers and arguments, and requiring the programmer to explicitly state the types. In the end, we generally decided that asking the programmer for more information was a better choice, as it made our jobs a bit easier. This helped eliminate the need for multiple passes through the program.

\subsection{Language terminals}

A few language terminals will be described here and used in the rest of the
language to construct most of the other features.

\begin{align}
    id &\rightarrow \textt{[a-zA-Z\_][a-zA-Z0-9\_]}\,\,\,\,\textt{(identifiers)}\\
    type &\rightarrow (i8|i16|i32|i64|u8|u16|u32|u64)*^*\,\,\,\,\textt{(types)} \\
\end{align}

\subsection{Expressions and Statements}

There are two kinds of code execution in Backbone: Statements and expressions.
Statements are a standalone piece of execution that executes only for its side effects.
Specifically, input, output, and let bindings are the most useful applications of
statements in Backbone. Expressions, on the other hand, are evaluated for their use in 
statements. Expressions simply reduce to values at runtime, and live to give meaning
to the side effects produced by statements.

Specifically, an "Expression", which will be used in the grammar definitions of
many structures throughout backbone, can be any of the following: A $\texttt{call}$, $\texttt{call-vargs}$, an arithmetic operator such as $+$ or $-$,
a comparison, a $\texttt{load}$, or itself just a value.

Statements, on the other hand, have a variety of meanings: Statements include let bindings,
return statements, conditional evaluations, calls and call-vargs, automatically deleted
declarations, and storing a value into a variable. Notice that both variadic and regular
calls are statements and expressions. The reason for this is that sometimes the value is not
needed for a function, and it was a nice language feature to be able to simply call a function
without having to use a let binding with an ignored variable name.

\subsection{Structs}

Users can define their own struct data types as a statement. The general
grammar is defined as follows:

\begin{align}
Struct &\rightarrow (\texttt{struct}\,id\,Fields) \\
Fields &\rightarrow Field\,Fields\,|\,Field \\
Field  &\rightarrow (id\,type)
\end{align}

An example of this syntax would be the following struct definition:

$\texttt{(struct SymbolicExpression}$

$\texttt{  (value i8*)}$

$\texttt{  (list SymbolicExpression**)}$

$\texttt{  (len u64)}$

$\texttt{  (cap u64))}$

Structs can be indexed using the $\texttt{(index} id type \texttt{Expression)}$ expression. Here, the
id is the identifier for the struct, type is the definition of the struct, and Expression is an expression
that evaluates to a $\texttt{i32}$ value. This expression returns the value that is stored inside the
struct given the specific index. In the future, we would like to support named variables in structs which
would allow for more code readability, but would not add to functionality.

Since a pointer to an element in a struct can be retrieved using $\textt{index}$, values can easily be
modified by using the $\texttt{store}$ keyword. Likewise, an element inside of a struct can be retrieved
using the $\texttt{load}$ keyword after indexing. The following is an example of the use of structs, followed
by the corresponding LLVM genereated by the compiler:

\footnotesize{ 
$\texttt{(struct Basic}$

$\texttt{  (a i32))}$

$\texttt{}$

$\texttt{(str-table}$

$\texttt{  (0 "Basic\{a = \%d\}\\0A\\00"))}$

$\texttt{}$

$\texttt{(decl calloc (types i64 i64) i8*)}$

$\texttt{}$

$\texttt{(def makeBasic (params (a i32)) Basic*}$

$\texttt{  (let r (cast i8* Basic* (call calloc (types i64 i64) i8* (args 1 8))))}$

$\texttt{  (store a i32 (index r Basic 0))}$

$\texttt{  (return r Basic*))}$

$\texttt{}$

$\texttt{(decl printf (types i8* ...) i32)}$

$\texttt{}$

$\texttt{(def main (params (argc i32) (argv i8**)) i32}$

$\texttt{  (let t (call makeBasic (types i32) Basic* (args 7)))}$

$\texttt{  (call-vargs printf (types i8* i32) i32 (args}$

$\texttt{    (str-get 0)}$

$\texttt{    (load i32 (index t Basic 0))))}$

$\texttt{  (return 0 i32))}$
}

\normalsize
And the corresponding LLVM bytecode:

\footnotesize{
$\texttt{\%struct.Basic = type \{ i32 \}}$

$\texttt{@str.0 = private unnamed\_addr constant [15 x i8] c"Basic\{a = \%d\}\\0A\\00", align 1}$

$\texttt{}$

$\texttt{declare i8* @calloc(i64, i64)}$

$\texttt{declare i32 @printf(i8*, ...)}$

$\texttt{define \%struct.Basic* @makeBasic(i32 \%a) \{}$

$\texttt{entry:}$
  
$\texttt{  \%\$0 = call i8* (i64, i64) @calloc(i64 1, i64 8)}$
  
$\texttt{  \%r = bitcast i8* \%\$0 to \%struct.Basic*}$
  
$\texttt{  \%\$1 = getelementptr inbounds \%struct.Basic, \%struct.Basic* \%r, i32 0, i32 0}$
  
$\texttt{  store i32 \%a, i32* \%\$1}$
  
$\texttt{  ret \%struct.Basic* \%r}$

$\texttt{\}}$

$\texttt{}$

$\texttt{define i32 @main(i32 \%argc, i8** \%argv) \{}$

$\texttt{entry:}$
  
$\texttt{  \%t = call \%struct.Basic* (i32) @makeBasic(i32 7)}$
  
$\texttt{  \%\$2 = getelementptr inbounds \%struct.Basic, \%struct.Basic* \%t, i32 0, i32 0}$
  
$\texttt{  \%\$1 = load i32, i32* \%\$2}$
  
$\texttt{  \%\$0 = call i32 (i8*, ...) @printf(i8* getelementptr inbounds ([15 x i8], [15 x i8]* @str.0, i64 0, i64 0),
i32 \%\$1)}$

$\texttt{  ret i32 0}$

$\texttt{\}}$
}

\normalsize

Which has the expected output of $\texttt{Basic\{a = 7\}}$.

\subsection{Calls}

Backbone supports statically typed functions that can be recursive and nested. Splitting code into
subroutines or functions is incredibly important in software engineering, and are vital to effectively
implementing system calls.

The general grammar is as follows:

$$\texttt{call}\,\,\,id\,\,\,(\texttt{types}\,type^*)\,\,\,(\texttt{args}\,\,\,Expression^*)\,\,\,type$$

Call is one of the more interesting features since it is both a statement and an expression.
The implications of this is that we can have nested calls. Since LLVM requires one expression evaluation
per instruction, calls with nested expressions need to be evaluated one at a time. We do this by
making a pass over the syntax tree and flattening expressions. This is further discussed in our section
on flattening.

An example of this syntax would be the following call statement:

$\texttt{(call printSexp void ((program Sexp*) (0 i32)))}$

\subsection{Qualified Types}

Converting between backbone and LLVM types is fairly simple. If the type is a user defined struct,
the LLVM type replaced is simply $\%struct.$ concatenated with the user defined struct type. If it
is a primitive type, or a pointer to a primitive type, they have the same representation. Here
are two examples of qualified types:

\begin{align}
\texttt{i32***}\quad &\rightarrow \quad \texttt{i32***} \\
\texttt{SymbolicExpression**}\quad &\rightarrow\quad \texttt{\%struct.SymbolicExpression**}
\end{align}

\subsection{Let}


\subsection{Conditionals}

One of the most basic functionalities of programming languages is the ability of the program to branch, typically completed with some sort of condition-checking. Backbone, like many other languages, uses the \texttt{if} keyword to indicate this. After the \texttt{if} keyword, the programmer should have a predicate, and then an expression that should be evaluated if the predicate evaluates to a non-zero number.

The general grammar is as follows:

$\texttt{if (predicate)}$

$\texttt{    (statement}^+\texttt{)}$

The following syntax example has a predicates that compare two numbers, the syntax of which will described shortly: (IMPROVE)

$\texttt{(if (< i32 2 1)}$

$\texttt{     (call puts (types i8*) i32 (args (str-get 0))))}$

$\texttt{  (if (>= i32 argc 3)}$

$\texttt{     (call puts (types i8*) i32 (args (str-get 1))))}$

\subsection{Return}

Of course, it is important to be able to return from a function call with useful information, so that the result of the function can be used elsewhere.
Our return simply requires the programmer to use the \textt{return} keyword, followed by the value to return and then its return type. This return type should match the return type of the function that was declared.
The general grammar is as follows:

$\texttt{return expression}\quad type$

An example of this syntax would be the following:

$\texttt{(return i i32)}$

\subsection{Declarations}
\subsection{Function Definitions}

As mentioned, Backbone supports function definitions, useful for separating code into reusable chunks and making programming similar tasks much simpler. Despite being a more challenging feature to support, we all agreed that functions were completely necessary for this project. 

Firstly, to define the name, return type, and arguments of the function, the programmer should use the \texttt{def} keyword, followed by the name of the function. Next, the parameters should be listed after a \texttt{params} keyword, each with the id and type of the argument. The return type is then added on after, and finally, the body of the function. This should end in a return statement, such that the return statement's type matches the return type of the function definitions.
The general syntax is as follows:

$\texttt{(def }\,\,\,id\,\,\,\texttt{(params (}\,\,\,id\,\,\,\,\,\,type\,\,\,\texttt{))}\,\,\,type\,\,\,\texttt{(statement}^+\texttt{))}$


The following syntax creates a function called \texttt{func} with two arguments, \texttt{a} of type 32-bit int, and \texttt{b} also of type 32-bit int. \texttt{func} simply returns the value of its argument a:

$\texttt{(def func (params (a i32) (b i32)) i32}$

$\texttt{  (return a i32))}$

\subsection{Binary Operations} 
\subsubsection{Comparators}
structure:
$$\,\,\,comparator\,\,\, \texttt{expression } \texttt{expression}$$

examples: 
$\texttt{(< i32 3 2)}$
$\texttt{(< i32 x y)}$$

\section{Conclusion}

\end{document}
